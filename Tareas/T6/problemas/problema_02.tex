\documentclass[../main.tex]{subfiles}

\begin{document}
\begin{problema}[3][El problema del pizzero]
	Un pizzero, después de amasar su masa para una pizza, extiende la masa
	de tal forma que consigue formar un disco delgado de grosor \(h\) y
	radio \(R\). Después, el pizzero gira este disco a una velocidad angular
	\(\omega\) respecto al eje de simetría del disco. Desde el sistema
	del laboratorio, no hay fuerzas de cuerpo.

	\begin{enumerate}
		\item Si se forma como referencia del sistema que gira con el disco,
		      aparece la fuerza centrífuga. Escribir la densidad de fuerza
		      centrífuga en coordenadas cilíndricas y obtener \(\vect{P}\).
		\item Obtener el vector de desplazamiento \(\vect{u} = u_{r}(r)\uvec{e}_{r}\)
		      aplicando las condiciones de frontera apropiadas.
	\end{enumerate}
\end{problema}
\end{document}
