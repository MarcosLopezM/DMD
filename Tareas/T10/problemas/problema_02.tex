\documentclass[../main.tex]{subfiles}

\begin{document}
\begin{problema}[3]
	Un fluido con densidad \(\rho\) y viscosidad \(\mu\) fluye dentro de un tubo
	largo y cónico con longitud \(L\) y radio \(R(x) = \bigl(1 - \alpha x/L\bigr)R_{0}\)
	como se muestra en la \zcref{fig:mass-flow-conical-tube} \(\alpha\) es el ángulo que forma la pared
	del cono con el eje \(x\), \(R_{0}\) es el radio máximo del tubo.
	Considerar que \(\alpha < 1\) y \(R_{0} \ll L\). Obtener el flujo
	de masa \(Q\) a través del tubo, dada una diferencia de presión
	\(\avg{p}\) entre la entrada y salida del tubo.

	\begin{figure}[htb]
		\centering
		\includegraphics[width=0.49\textwidth]{T10-p-000}
		\caption[Pie de figura para el problema 1]{}
		\label{fig:mass-flow-conical-tube}
	\end{figure}
\end{problema}
\end{document}
