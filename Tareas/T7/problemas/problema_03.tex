\documentclass[../main.tex]{subfiles}

\begin{document}
\begin{problema}
	El campo de velocidad \(\bm{v}\) se dice que es irrotacional cuando
	la vorticidad es cero, es decir, \(\bm{w} = 0\). Entonces,
	existe un potencial de velocidad \(\phi(\bm{x}, t)\) tal que
	\(\bm{v} = \nabla{\phi}\). Muestra que las ecuaciones de
	Navier-Stokes del problema anterior se pueden expresar como:

	\begin{equation*}
		\rho \nabla{\Biggl(\pdv{\phi}{t} + \dfrac{1}{2}\Bigl(\nabla{\phi}\Bigr)^{2}\Biggr)} =
		\rho \bm{f} - \nabla{p} + (\lambda + 2\mu)\nabla{(\nabla^{2}{\phi})}.
	\end{equation*}
\end{problema}
\end{document}
