\documentclass[../main.tex]{subfiles}

\begin{document}
\begin{problema}
	Supón que el tensor de esfuerzos \(\bm{\sigma}\) es simétrico,
	es decir, \(\bm{\sigma}^{T} = \bm{\sigma}\). Entonces:

	\begin{enumerate}
		\item Muestra que la ecuación de energía se puede escribir como:

		      \begin{equation*}
			      \rho \mdv*{\Bigl(e + \dfrac{v^{2}}{2}\Bigr)}{t} =
			      \nabla \cdot{(\bm{\sigma} \cdot \bm{v})} + \rho \bm{f} \cdot \bm{v} +
			      \rho r_{h} - \nabla \cdot{\bm{q}}.
		      \end{equation*}

		\item Obtén la versión termodinámica de la ecuación de energía:

		      \begin{equation*}
			      \rho\mdv{e}{t} = \nabla \cdot{(\bm{\sigma} \cdot \bm{v})} -
			      \nabla \cdot{\bm{v} \cdot \bm{\sigma}} + \rho r_{h} -
			      \nabla \cdot{\bm{q}}.
		      \end{equation*}
	\end{enumerate}
\end{problema}
\end{document}
