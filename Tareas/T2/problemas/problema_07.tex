\documentclass[../main.tex]{subfiles}

\begin{document}
\begin{problema}
	Muestra que la ecuación característica para un tensor simétrico de
	segundo orden \(\vb{\Phi}\) se puede expresar como:

	\begin{equation*}
		-\lambda^{3} + I_{1}\lambda^{2} - I_{2}\lambda + I_{3} = 0.
	\end{equation*}

	Donde

	\begin{align*}
		I_{1} & = \phi_{kk},\qquad I_{2} = \tfrac{1}{2}(\phi_{ii}\phi_{jj} - \phi_{ij}\phi_{ji}), \\
		I_{3} & = \tfrac{1}{6}(2\phi_{ij}\phi_{jk}\phi_{ki} - 3\phi_{ij}\phi_{ji}\phi_{kk}
		+ \phi_{ii}\phi_{jj}\phi_{kk}) = \det(\phi_{ij}).
	\end{align*}
\end{problema}
\end{document}
