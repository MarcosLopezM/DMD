\documentclass[../main.tex]{subfiles}

\begin{document}
\begin{problema}[3]
	Una cubeta con agua gira alrededor de su eje de simetría con una velocidad
	angular constante \(\vect{\omega}\), como se muestra en la \zcref{fig:bucket-rotating}.
	Demostrar que la superficie del líquido se curva, tomando la forma de
	un paraboloide de revolución. Suponer que el agua tiene una densidad \(\rho\)
	constante. Considerar la gravedad como constante y apuntando en la
	dirección negativa del eje \(z\).

	\begin{figure}[htb]
		\centering
		\includegraphics[scale=0.4]{T9-p-001.jpg}
		\caption[Pie de figura para el problema 4]{}
		\label{fig:bucket-rotating}
	\end{figure}
\end{problema}
\end{document}
