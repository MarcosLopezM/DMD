\documentclass[../main.tex]{subfiles}

\begin{document}
\begin{problema}[2]
	Considerar un flujo Newtoniano incompresible en torno a un barco. El flujo
	es estacionario medido desde el barco. La ecuación de Navier-Stokes que
	describe el flujo es

	\begin{equation*}
		\rho \mdv{\vect{v}}{t} = - \nabla{p} + \mu \nabla^{2}{\vect{v}} + \rho \vect{g}.
	\end{equation*}

	Responder los siguientes incisos:

	\begin{enumerate}
		\item ¿Qué dice el teorema \(\Pi\) de Buckingham?
		\item  Encontrar las variables adimensionales que describen el flujo
		      alrededor del barco. Adimensionalizar la ecuación de Navier-Stokes
		      que describe el flujo. ¿Qué números adimensionales aparecen y qué
		      representa cada uno?
	\end{enumerate}
\end{problema}
\end{document}
