\documentclass[../main.tex]{subfiles}

\begin{document}
\begin{problema}[4]
	Determinar la deformación que sufre una esfera hueca, con radio interior \(R_{1}\)
	y radio exterior \(R_{2}\). La esfera está hecha de un material elástico lineal
	\(e\) isotrópo y tiene por constantes elásticas el módulo de Young \(Y\) y la
	razón de Poisson \(\nu\). Dentro de la esfera hay un gas en reposo con presión
	\(p_{1}\) y la esfera se encuentra en un gas (externo) con presión \(p_{2}\).
	Usar coordenadas esféricas para resolver el problema y supóngase que el
	campo vectorial de desplazamiento depende únicamente de la variable radial
	\(r\).

	\startsolution

	Para resolver este problema debemos encontrar soluciones para el vector
	desplazamiento que serán del tipo \(\vect{u} = u_{r}(r)\uvec{e}_{r}\) en
	coordenadas esféricas.

	La ecuación de equilibrio que nos interesa es:

	\begin{equation*}
		(\lambda + 2\mu)\nabla{(\nabla \cdot{\vect{u}})} - \mu(\nabla \mul{(\nabla \mul{\vect{u}})}) = 0.
	\end{equation*}

	Como sabemos que el desplazamiento es radial en todos lados y únicamente depende de \(r\),

	\begin{equation*}
		\nabla \mul{\vect{u}} = 0.
	\end{equation*}

	Por lo que,

	\begin{equation*}
		\nabla{(\nabla \cdot{\vect{u}})} = 0,
	\end{equation*}

	lo cual implica que \(\nabla \cdot{\vect{u}} = \text{cte}\).

	\begin{align*}
		\nabla \cdot{\vect{u}} & = \dfrac{1}{r^{2}}\pdv{r^{2}u_{r}}{r} + \dfrac{1}{r\sin\theta}\pdv{u_{\theta}\sin\theta}{\theta} + \dfrac{1}{r\sin\theta}\pdv{u_{\varphi}}{\varphi}, \\
		\Longrightarrow\;\pdv{r^{2}u_{r}}{r} = A
	\end{align*}

	Integrando,

	\begin{align}
		r^{2}u_{r} & = \dfrac{A}{3}r^{3} + B,\nonumber                \\
		u_{r}      & = \dfrac{A}{3}r + \dfrac{B}{r^{2}}.\label{eq:ur}
	\end{align}

	Calculamos ahora \(\nabla{(\nabla \cdot{\vect{u}})}\), cuya única componente es
	la radial, i.e.

	\begin{align*}
		\nabla{(\nabla \cdot{\vect{u}})} & = \pdv*{\Bigl[\dfrac{A}{3}r + \dfrac{B}{r^{2}}\Bigr]}{r}, \\
		\nabla{(\nabla \cdot{\vect{u}})} & = \Bigl(\dfrac{A}{3} - 2\dfrac{B}{r^{3}}\Bigr).
	\end{align*}

	El tensor de deformaciones es:

	\begin{equation*}
		E_{ij} = \dfrac{1}{2}\Bigl(\pdv{u_{i}}{x_{j}} + \pdv{u_{j}}{x_{i}}\Bigr).
	\end{equation*}

	Por lo que las componentes del tensor de deformaciones son:

	\begin{alignat*}{2}
		E_{rr}             & = \dfrac{A}{3} -2\dfrac{B}{r^{3}},\qquad E_{\theta\theta} & {}={} & \dfrac{A}{3} + \dfrac{B}{r^{3}}, \\
		E_{\varphi\varphi} & = \dfrac{A}{3} + \dfrac{B}{r^{3}},\qquad E_{r\theta}      & {}={} & E_{\theta r} = 0,                \\
		E_{r\varphi}       & = E_{\varphi r} = 0,\qquad E_{\theta\varphi}              & {}={} & E_{\varphi\theta} = 0.
	\end{alignat*}

	Ahora, las componentes del tensor de esfuerzos están dadas por:

	\begin{equation*}
		\sigma_{ij} = \lambda E_{kk}\delta_{ij} + 2\mu E_{ij}.
	\end{equation*}

	Así,

	\begin{align}
		\sigma_{rr} & = \lambda(E_{rr} + 2E_{\theta\theta}) + 2\mu E_{rr},\nonumber            \\
		\sigma_{rr} & = (\lambda + 2\mu)E_{rr} + 2\lambda E_{\theta\theta}.\label{eq:sigma_rr}
	\end{align}

	Notemos que en el término de la traza no aparece \(E_{\varphi\varphi}\), esto se debe a
	que \( E_{\theta\theta} = E_{\varphi\varphi}\).

	\begin{align}
		\sigma_{\theta\theta} & = \lambda(E_{rr} + 2E_{\theta\theta}) + 2\mu E_{\theta\theta},\nonumber                     \\
		\sigma_{\theta\theta} & = \lambda E_{rr} + 2(1 + \mu)E_{\theta\theta} = \sigma_{\varphi\varphi}.\label{eq:sigma_ff}
	\end{align}

	Escribiendo las \zcref{eq:sigma_rr, eq:sigma_ff} en términos del
	módulo de Young \(Y\) y la razón de Poisson \(\nu\) a través de las relaciones
	\(\lambda = \dfrac{Y\nu}{(1 + \nu)(1 - 2\nu)}\) y \(\mu = \dfrac{Y}{2(1 + \nu)}\),

	\begin{align}
		\sigma_{rr} & = \dfrac{Y}{(1 + \nu)(1 - 2\nu)}\Bigl[(1 - \nu)E_{rr} + 2\nu E_{\theta\theta}\Bigr],\nonumber                                                                           \\
		            & = \dfrac{Y}{(1 + \nu)(1 - 2\nu)}\Biggl[(1 - \nu)\Bigl(\dfrac{A}{3} - 2\dfrac{B}{r^{3}}\Bigr) + 2\nu \Bigl(\dfrac{A}{3} + \dfrac{B}{r^{3}}\Bigr)\Biggr],\nonumber        \\
		            & = \dfrac{Y}{(1 + \nu)(1 - 2\nu)}\Biggl[\dfrac{A}{3} - 2\dfrac{B}{r^{3}} - \dfrac{A\nu}{3} + 2\dfrac{B\nu}{r^{3}} + 2\dfrac{A\nu}{3} + 2\dfrac{B\nu}{3}\Biggr],\nonumber \\
		            & = \dfrac{Y}{(1 + \nu)(1 - 2\nu)}\Biggl[(1 + \nu)\dfrac{A}{3} - 2(1 - 2\nu)\dfrac{B}{r^{3}}\Biggr],\nonumber                                                             \\
		\sigma_{rr} & = \dfrac{Y}{1 - 2\nu}\dfrac{A}{3} - \dfrac{2Y}{1 + \nu}\dfrac{B}{r^{3}}.\label{eq:sigma_rr-young-poisson}
	\end{align}

	Usando las condiciones de frontera \(\sigma(r = R_{1}) = -p_{1}\) y \(\sigma_{rr}(r = R_{2}) = -p_{2}\). Así,

	\begin{align}
		\sigma_{rr}(r = R_{1}) = -p_{1} & = \dfrac{Y}{1 - 2\nu}\dfrac{A}{3} - \dfrac{2Y}{1 + \nu}\dfrac{B}{R_{1}^{3}},\nonumber                                                                              \\
		A                               & = \dfrac{3(1 - 2\nu)}{Y}\Bigl(\dfrac{2Y}{1 + \nu}\dfrac{B}{R_{1}^{3}} - p_{1}\Bigr).\label{eq:coeff-A},                                                            \\
		\sigma_{rr}(r = R_{2}) = -p_{2} & = \dfrac{Y}{1 - 2\nu}\dfrac{A}{3} - \dfrac{2Y}{1 + \nu}\dfrac{B}{R_{2}^{3}},\nonumber                                                                              \\
		-p_{2}                          & = \dfrac{Y}{3(1 - 2\nu)}\Biggl[\dfrac{3(1 - 2\nu)}{Y}\Bigl(\dfrac{2Y}{1 + \nu}\dfrac{B}{R_{1}^{3}} - p_{1}\Bigr)\Biggr] - \dfrac{2Y}{1 + \nu}\dfrac{B}{R_{2}^{3}}, \\
		- p_{2}                         & = \dfrac{2Y}{1 + \nu}\Bigl(\dfrac{1}{R_{1}^{3}} - \dfrac{1}{R_{2}^{3}}\Bigr)B - p_{1},\nonumber                                                                    \\
		B                               & = \dfrac{(1 + \nu)R_{1}^{3}R_{2}^{3}}{2Y(R_{2}^{3} - R_{1}^{3})}(p_{1} - p_{2}).\label{eq:coeff-B}
	\end{align}

	Sustituyendo la \zcref{eq:coeff-B} en la \zcref{eq:coeff-A},

	\begin{align}
		A & = \dfrac{3(1 - 2\nu)}{Y}\Biggl[\dfrac{2Y}{1 + \nu}\dfrac{1}{R_{1}^{3}}\Biggl(\dfrac{(1 + \nu)R_{1}^{3}R_{2}^{3}}{2Y(R_{2}^{3} - R_{1}^{3})}(p_{1} - p_{2})\Biggr) - p_{1}\Biggr],\nonumber \\
		  & = \dfrac{3(1 - 2\nu)}{Y}\Biggl[\dfrac{R_{2}^{3}}{R_{2}^{3} - R_{1}^{3}}(p_{1} - p_{2}) - p_{1}\Biggr],\nonumber                                                                            \\
		  & = \dfrac{3(1 - 2\nu)}{Y}\Biggl[\dfrac{R_{2}^{3}p_{1} - R_{2}^{3}p_{2} - R_{2}^{3}p_{1} + R_{1}^{3}p_{1}}{R_{2}^{3} - R_{1}^{3}}\Biggr],\nonumber                                           \\
		A & = \dfrac{3(1 - 2\nu)}{Y}\Biggl[\dfrac{R_{1}^{3}p_{1} - R_{2}^{3}p_{2}}{R_{2}^{3} - R_{1}^{3}}\Biggr].\label{eq:coeff-A-final}
	\end{align}

	Sustituyendo las \zcref{eq:coeff-B,eq:coeff-A-final} en la \zcref{eq:ur}, obtenemos la solución final para el desplazamiento radial:

	\begin{empheq}[box = \mainresult]{equation*}
		u_{r}(r) = \dfrac{1 - 2\nu}{Y}\Biggl[\dfrac{R_{1}^{3}p_{1} - R_{2}^{3}p_{2}}{R_{2}^{3} - R_{1}^{3}}\Biggr] + \dfrac{(1 + \nu)R_{1}^{3}R_{2}^{3}}{2Y(R_{2}^{3} - R_{1}^{3})}(p_{1} - p_{2})\dfrac{1}{r^{2}}.
	\end{empheq}

\end{problema}
\end{document}
