\documentclass[../main.tex]{subfiles}

\begin{document}
\begin{problema}[4]
	Determinar la deformación que sufre una esfera hueca, con radio interior \(R_{1}\)
	y radio exterior \(R_{2}\). La esfera está hecha de un material elástico lineal
	\(e\) isotrópo y tiene por constantes elásticas el módulo de Young \(Y\) y la
	razón de Poisson \(\nu\). Dentro de la esfera hay un gas en reposo con presión
	\(p_{1}\) y la esfera se encuentra en un gas (externo) con presión \(p_{2}\).
	Usar coordenadas esféricas para resolver el problema y supóngase que el
	campo vectorial de desplazamiento depende únicamente de la variable radial
	\(r\).
\end{problema}
\end{document}
