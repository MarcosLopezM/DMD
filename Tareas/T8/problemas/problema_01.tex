\documentclass[../main.tex]{subfiles}

\begin{document}
\begin{problema}
	Muestra que la \(\mdv{\bm{\mathrm{v}}}{t}\) también puede escribirse como

	\begin{equation*}
		\mdv{\bm{\mathrm{v}}}{t} = \pdv{\bm{\mathrm{v}}}!{t} + \nabla{\Bigl(\dfrac{v^{2}}{2}\Bigr)}
		- \bm{\mathrm{v}} \mul (\nabla \mul{\bm{\mathrm{v}}}),\quad v^{2} = \bm{\mathrm{v}} \cdot \bm{\mathrm{v}}
	\end{equation*}

	\startsolution

	La derivada material es

	\begin{equation}
		\mdv{\vect{v}}{t} = \pdv{\vect{v}}{t} + \vect{v} \cdot \nabla{\vect{v}}.
		\label{eq:mdv-p1}
	\end{equation}

	Para a llegar a la forma deseada usamos la siguiente identidad vectorial

	\begin{align*}
		\nabla{(v^{2})}                      & = \dotprod{\vect{v}}{\nabla{\vect{v}}} + \dotprod{\vect{v}}{\nabla{\vect{v}}} + \cprod{\vect{v}}{( \cprod{\nabla}{\vect{v}} )}
		+ \cprod{\vect{v}}{( \cprod{\nabla}{\vect{v}} )},                                                                                                                     \\
		                                     & = 2 \Bigl[\dotprod{\vect{v}}{\nabla{\vect{v}}} + \vect{v}(\cprod{\nabla}{\vect{v}})\Bigr],                                     \\
		\dotprod{\vect{v}}{\nabla{\vect{v}}} & = \nabla{\Bigl(\dfrac{v^{2}}{2}\Bigr)} -
		\cprod{\vect{v}}{( \cprod{\nabla}{\vect{v}} )}.
	\end{align*}

	Sustituyendo este resultado en la \zcref{eq:mdv-p1},

	\begin{empheq}[box = \mainresult]{equation}
		\mdv{\vect{v}}{t} = \pdv{\vect{v}}{t} + \nabla{\Bigl(\dfrac{v^{2}}{2}\Bigr)} -
		\cprod{\vect{v}}{( \cprod{\nabla}{\vect{v}} )}.
		\label{eq:mdv-result-p1}
	\end{empheq}
\end{problema}
\end{document}
