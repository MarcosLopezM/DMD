\documentclass[../main.tex]{subfiles}

\begin{document}
\begin{problema}
	El movimiento de un cuerpo continuo está dado por

	\begin{align*}
		x_{1} & = \tfrac{1}{2}(X_{1} + X_{2})\mathrm{e}^{t} + \tfrac{1}{2}(X_{1} - X_{2})\mathrm{e}^{-t}, \\
		x_{2} & = \tfrac{1}{2}(X_{1} + X_{2})\mathrm{e}^{t} + \tfrac{1}{2}(X_{1} - X_{2})\mathrm{e}^{-t}, \\
		x_{3} & = X_{3},
	\end{align*}

	para \(0 \leq t < \infty\). Determina

	\begin{enumerate}
		\item velocidad en la descripción material,
		\item velocidad en la descripción espacial,
		\item componentes de los tensores de tasa de deformación
		      y de vorticidad.
	\end{enumerate}
\end{problema}
\end{document}
