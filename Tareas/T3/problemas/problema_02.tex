\documentclass[../main.tex]{subfiles}

\begin{document}
\begin{problema}
	El movimiento de un cuerpo se describe por el siguiente mapeo

	\begin{equation*}
		\chi(\bm{X}) = (X_{1} + t^{2}X_{2})\ev{1} +
		(X_{2} + t^{2}X_{1})\ev{2} +
		X_{2}\ev{3},\\quad
		0 \leq t < \infty.
	\end{equation*}

	Determina

	\begin{enumerate}
		\item los componentes del gradiente de deformación \(\bm{F}\) y su inversa,
		\item los componentes del desplazamiento, velocidad y aceleración,
		\item Esboza la forma del cuerpo deformado a los tiempos
		      \(t = 0,1,2,3\), asumiendo que originalmente era un cubo unitario.
	\end{enumerate}
\end{problema}
\end{document}
