\documentclass[../main.tex]{subfiles}

\begin{document}
\begin{problema}
	Un cuerpo deformable, plano, originalmente circular, experimenta
	una deformación, de modo que al alcanzar el equilibrio exhibe la
	forma de una elipse cuyos ejes de simetría coinciden con los ejes
	coordenados cartesianos.

	\begin{enumerate}
		\item ¿Cuál es la transformación de coordenadas que mapea un
		      círculo en una elipse? Haz un esquema del círculo y su
		      deformación.
		\item Calcula el jacobiano y el jacobiano inverso de esta
		      transformación.
		\item Si la deformación es pequeña (es decir, si aproximamos linealmente).
		      ¿Cuál es la forma de la matriz que representa al tensor de deformación?
	\end{enumerate}
\end{problema}
\end{document}
