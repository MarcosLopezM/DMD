\documentclass[../main.tex]{subfiles}

\begin{document}
\begin{problema}
	Un cuerpo deformable, plano, originalmente circular, experimenta
	una deformación, de modo que al alcanzar el equilibrio exhibe la
	forma de una elipse cuyos ejes de simetría coinciden con los ejes
	coordenados cartesianos.

	\begin{enumerate}
		\item ¿Cuál es la transformación de coordenadas que mapea un
		      círculo en una elipse? Haz un esquema del círculo y su
		      deformación.
		\item Calcula el jacobiano y el jacobiano inverso de esta
		      transformación.
		\item Si la deformación es pequeña (es decir, si aproximamos linealmente).
		      ¿Cuál es la forma de la matriz que representa al tensor de deformación?
	\end{enumerate}
	\startsolution

	\section{Inciso (a)}
	Los puntos del plano circular centrado en el origen, están dados por:

	\begin{equation*}
		X^{2} + Y^{2} = R^{2}.
	\end{equation*}

	Y sabemos que la elipse es un reescalamiento de éste, por lo que, el mapeo es:

	\begin{equation}
		\begin{aligned}
			x & = aX, \\
			y & = bY, \\
			z & = Z.
		\end{aligned}
		\label{eq:mapping}
	\end{equation}

	\section{Inciso (b)}

	Para obtener los jacobianos de la transformación, primero debemos calcular el vector
	de desplazamiento \(\vect{u}\),

	\begin{equation*}
		\vect{u} = \vect{x} - \vect{X},
	\end{equation*}

	tal que

	\begin{align*}
		\vect{u} & = \Bigl(aX - X\Bigr)\eu{x} + \Bigl(bY - Y\Bigr)\eu{y} + (Z - Z)\eu{z}, \\
		\vect{u} & = \Bigl(a - 1\Bigr)X\eu{x} + \Bigl(b - 1\Bigr)Y\eu{y}.
	\end{align*}

	Por lo que el jacobiano de la transformación es

	\begin{empheq}[box = \mainresult]{equation}
		\nabla{\vect{u}} =
		\begin{pNiceMatrix}
			a - 1 & 0     & 0 \\
			0     & b - 1 & 0 \\
			0     & 0     & 0 \\
		\end{pNiceMatrix}
		\label{eq:jacobian}
	\end{empheq}

	Mientras que el jacobiano inverso de la transformación no se puede calcular, ya que
	el determinante de la \zcref{eq:jacobian} es cero.

	\section{Inciso (c)}

	Para deformaciones pequeñas sabemos que el tensor de deformación es

	\begin{equation*}
		\dbloverline{E} = \nabla{\vect{u}}.
	\end{equation*}

	Entonces,

	\begin{empheq}[box = \mainresult]{equation*}
		\dbloverline{E} =
		\begin{pNiceMatrix}
			a - 1 & 0     & 0 \\
			0     & b - 1 & 0 \\
			0     & 0     & 0 \\
		\end{pNiceMatrix}
	\end{empheq}
\end{problema}
\end{document}
