%! TeX program = lualatex
\documentclass[../main.tex]{subfiles}

\begin{document}
\begin{problema}
	Sea la base

	\begin{equation*}
		\ev{1} = -\uveci - \uvecj,\quad \ev{2} = \uveci + 2\uvecj - 2\uveck,
		\quad \ev{3} = 2\uveci + \uvecj + \uveck,
	\end{equation*}

	donde los vectores \((\uveci, \uvecj, \uveck)\) forman una base ortonormal. Determina

	\begin{enumerate}
		\item La base recíproca \((\evc{1}, \evc{2}, \evc{3})\) en términos de
		      la base ortonormal \((\uveci, \uvecj, \uveck)\),
		      \startsolution

		      Recordamos que la base recíproca en términos de la base original está dada por

		      \begin{equation*}
			      \vect{e_{1}} = \dfrac{\crossp{\vect{e_{2}}}{\vect{e_{3}}}}{V},\qquad
			      \vect{e_{2}} = \dfrac{\crossp{\vect{e_{3}}}{\vect{e_{1}}}}{V},\qquad
			      \vect{e_{3}} = \dfrac{\crossp{\vect{e_{1}}}{\vect{e_{2}}}}{V},
		      \end{equation*}

		      con \(V = \dotp{\vect{e_{3}}}{\crossp{\vect{e_{1}}}{\vect{e_{1}}}}\)

		      Calculando \(V\),

		      \begin{align*}
			      V            & = \vect{e_{3}} \cdot
			      \begin{vNiceMatrix}[baseline=2]
				      \uveci & \uvecj & \uveck \\
				      -1     & -1     & 0      \\
				      1      & 2      & -2
			      \end{vNiceMatrix},                                                        \\
			                   & = \vect{e_{3}} \cdot [\uveci(2 - 0) - \uvecj(2 - 0) + \uveck(-2 + 1)], \\
			                   & = (2, 1, 1) \cdot (2, -2, -1),                                         \\
			      \Aboxedsec{V & = 1.}
		      \end{align*}


		      El recíproco de \(\vec{e}^{1}\) queda

		      \begin{align*}
			      \evc{1}             & =
			      \begin{vNiceMatrix}
				      \uveci & \uvecj & \uveck \\
				      1      & 2      & -2     \\
				      2      & 1      & 1
			      \end{vNiceMatrix} =
			      \uveci(2 + 2) - \uvecj(1 + 4) + \uveck(1 - 4),        \\
			      \Aboxedmain{\evc{1} & = 4\uveci - 5\uvecj - 3\uveck.}
		      \end{align*}

		      Análogamente,


		      \begin{align*}
			      \Rightarrow\; \evc{2} & =
			      \begin{vNiceMatrix}
				      \uveci & \uvecj & \uveck \\
				      2      & 1      & 1      \\
				      -1     & -1     & 0
			      \end{vNiceMatrix} =
			      \uveci(0 + 1) - \uvecj(0 + 4) + \uveck(-2 + 1),        \\
			      \Aboxedmain{\evc{2}   & = \uveci - \uvecj - \uveck,}   \\
			      \Rightarrow\; \evc{3} & =
			      \begin{vNiceMatrix}
				      \uveci & \uvecj & \uveck \\
				      -1     & -1     & 0      \\
				      1      & 2      & -2
			      \end{vNiceMatrix} =
			      \uveci(2 - 0) - \uvecj(2 - 0) + \uveck(-2 + 1),        \\
			      \Aboxedmain{\evc{3}   & = 2\uveci - 2\uvecj - \uveck.}
		      \end{align*}

		\item Las magnitudes \(\nev{1}, \nev{2}, \nev{3}, \nevc{1}, \nevc{2}\) y \(\nevc{3}\),

		      La magnitud de los vectores de la base original es:

		      \begin{empheq}[box = \mainresult]{align*}
			      \nev{1} & = \sqrt{(-1)^{2} + (-1)^{2}} = \sqrt{2},         \\
			      \nev{2} & = \sqrt{(1)^{2} + (2)^{2} + (-2)^{2}} = 3,       \\
			      \nev{3} & = \sqrt{(2)^{2} + (1)^{2} + (1)^{2}} = \sqrt{6}.
		      \end{empheq}

		      Mientras que la magnitud de los vectores de la base recíproca son

		      \begin{empheq}[box = \mainresult]{align*}
			      \nev{1} & = \sqrt{(4)^{2} + (-5)^{2} + (-3)^{2}} = \sqrt{50},         \\
			      \nev{2} & = \sqrt{(1)^{2} + (-1)^{2} + (-1)^{2}} = \sqrt{3},       \\
			      \nev{3} & = \sqrt{(2)^{2} + (-2)^{2} + (-1)^{2}} = 3.
		      \end{empheq}

		\item Los componentes covariantes \(v_{1}, v_{2}\) y \(v_{3}\) de un vector
		      \(\vect{v}\) si sus componentes contravariantes están dados por
		      \(v^{1} = 1, v^{2} = 2, v^{3} = 3\).

		      \startsolution

		      Para determinar las componentes covariantes del vector \(\vect{v}\)
		      recordamos que están dadas por

		      \begin{equation*}
			      v_{i} = \delta_{ij}v^{j}.
		      \end{equation*}

		      Entonces,

		      \begin{align*}
			      \Rightarrow\; v_{1} & = \delta_{11}v^{1} + \delta_{12}v^{2} + \delta_{13}v^{3}, \\
			      \Aboxedmain{v_{1}   & = 1.}                                                     \\
			      \Rightarrow\; v_{2} & = \delta_{21}v^{1} + \delta_{22}v^{2} + \delta_{23}v^{3}, \\
			      \Aboxedmain{v_{2}   & = 1.}                                                     \\
			      \Rightarrow\; v_{1} & = \delta_{31}v^{1} + \delta_{32}v^{2} + \delta_{33}v^{3}, \\
			      \Aboxedmain{v_{3}   & = 1.}
		      \end{align*}
	\end{enumerate}
\end{problema}
\end{document}
