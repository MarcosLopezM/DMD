\documentclass[../main.tex]{subfiles}

\begin{document}
\begin{problema}
	La siguiente lista presenta algunos temas de estudio que pueden ser de
	interés para diferentes áreas de la física. Elige uno y evalúa que tan
	buena idea sería usar la \emph{hipótesis del continuo} para abordar ese
	problema. El formato es libre, aunque se recomienda no exceder una
	cuartilla incluyendo referencias.

	\begin{itemize}[twocol]
		\item Tormenta de arena.
		\item Nube interestelar.
		\item Flujo sanguíneo en seres vivos.
		\item Gases en la atmósfera superior.
		\item Materiales porosos.
		\item Oleaje en el océano.
		\item Formación y propagación de una avalancha de nieve.
		\item Metales líquidos.
		\item  Nubes atmosféricas.
		\item Flujo de tráfico vehicular.
	\end{itemize}

	\startsolution

	La hipótesis del continuo asume que la materia puede modelarse como un medio
	continuo, ignorando su estructura discreta a nivel molecular.
	Es decir, se considera a un fluido como un medio
	que ocupa todo el espacio disponible de manera que no existen huecos
	o espacios vacíos. Este enfoque puede aplicarse siempre y cuando
	la porción de materia no sea demasiado pequeña como para poner
	en manifiesto sus características moleculares.

	Esto resulta útil en el estudio del flujo vehicular. Aunque a nivel individual
	cada vehículo es una entidad discreta, al considerarse un número suficientemente
	grande de vehículos, el tráfico puede modelarse como un fluido mediante ecuaciones
	hiperbólicas, que dependen únicamente de condiciones iniciales y de frontera. Estos
	modelos también son conocidos como modelos macroscópicos, ya que permiten estudiar
	el comportamiento colectivo del tráfico en términos de variables agregadas como la
	densidad y el flujo.
\end{problema}
\end{document}
