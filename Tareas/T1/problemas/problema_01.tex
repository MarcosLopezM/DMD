\documentclass[../main.tex]{subfiles}

\begin{document}
\begin{problema}
	La siguiente lista presenta algunos temas de estudio que pueden ser de
	interés para diferentes áreas de la física. Elige uno y evalúa que tan
	buena idea sería usar la \emph{hipótesis del continuo} para abordar ese
	problema. El formato es libre, aunque se recomienda no exceder una
	cuartilla incluyendo referencias.

	\begin{itemize}
		\item Tormenta de arena.
		\item Nube interestelar.
		\item Flujo sanguíneo en seres vivos.
		\item Gases en la atmósfera superior.
		\item Materiales porosos.
		\item Oleaje en el océano.
		\item Formación y propagación de una avalancha de nieve.
		\item Metales líquidos.
		\item  Nubes atmosféricas.
		\item Flujo de tráfico vehicular.
	\end{itemize}
\end{problema}
\end{document}
